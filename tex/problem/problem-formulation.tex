\sectionLabel{Problem Formulation}

We treat \textit{\projectTitle{}} as a task of word disambiguation. The
ambiguity among words is modelled by \textit{confusion sets}. A confusion set
\(C = \{w_1, ..., w_n\}\) means that each word \(w_i\) in the set is ambiguous
with each other word \(w_j\) in the set \(C\). Thus if \(C = \{desert,
dessert\}\), then when the spelling-correction program ses an occurrence of
either \(desert\) or \(dessert\) in the target document, it takes it to be
ambiguous between \(desert\) and \(dessert\), and tries to infer from the
context which of the two it should be.

This treatment requires a collection of confusion sets to start with. There are
several ways to obtain such a collection. One is based on finding words in the
dictionary that are one typo away from each other \cite{Mays1991517}. Another finds
words that have the same or similar pronunciations. Since this was not the
focus of the work we did, we simply took (most of) our confusion sets from the
list of ``Common Errors in English Usage'' from the list curated by Paul Brians
\cite{website:paulbrians}.

A final point concerns the two types of errors a spelling-correction program
can make: false negatives (complaining about a correct word), and false
positives (failing to notice an error). We will make the simplifying assumption
that both kind of errors are equally bad. In practice, however, false negatives
are much worse, as users get irritated by the programs that badger them with
bogus complaints. However, given the probabilistic nature of the methods that
will be presented below, it would not be hard to modify them to take this into
account. We would merely set a confidence threshold, and report a suggested
correction only if the probability of the suggested word exceeds the
probability of the user's original spelling by at least a threshold amount. The
reason this was not done in the work reported here is that setting the
confidence threshold involves a certain subjective factor (which depends on the
user's ``irritability threshold''). Our simplifying assumption allows us to
measure performance objectively, by the single parameter of prediction
accuracy.
