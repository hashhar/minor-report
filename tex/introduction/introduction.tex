\chapterLabel{Introduction}

Writing is one of the predominant and vital ways of communication through which
humans can express their views to others and keep a record. It is one of the
most effective forms of language representation. It does the task of
representing language by engraving signs and symbols. Writing is composed of
vocabulary, semantics and grammar. Text is considered as the outcome of
writing.

The scalability of writing text has increased due to the availability of
computers, due to which many issues such as spelling mistakes have also
evolved. Mistakes can sidetrack readers from the efforts the writer has put in
his writing. Therefore it becomes important to remove these mistakes. Hence, it
prompted the need to use spelling checkers so that errors can be minimized
while writing. Spell checkers are either part of large applications, for
instance, search engines, email clients etc.\ or a standalone application that
is efficient in performing correction on a piece of text. Nearly all word
processors have a built-in spelling checker that flags the spelling mistakes.
It also provides the solution to correct these spelling mistakes by choosing a
possible alternative from a given list. For identification of spelling
mistakes, most spell checkers check each word drawn separately from the written
text against the dictionary-stored words. If the word is found while searching
the dictionary, it is considered as correct word regardless of its context.
This approach is efficient for identifying the non-word spelling mistakes but
other mistakes cannot be identified using this method. The other mistakes are
real-word spelling mistakes.

Real-word spelling mistakes are words that are correctly spelled but are not
intended by the user. Mistakes under this category go unrecognized by most
spell checkers because they handle non-word spelling mistakes by checking
against the dictionary word list only. This technique is effective to identify
the non-word spelling mistakes but not the real-word spelling mistakes. To
identify the real-word spelling mistakes, there is a need to utilize the
neighboring contextual information of the target word. An ex of such a
sentence is \ex{``I want to eat a peace of cake''} and the confused word
set in this case is \ex{\{peace, piece\}}. To identify that
\ex{`peace'} cannot be used in this case, we utilize the neighboring
contextual information \ex{`cake'} for word \ex{`piece'}.
