\chapterLabel*{Abstract}

\textit{\projectTitle{}} is the task of fixing spelling errors that result in
valid words, such as \textbfit{I'd like to eat desert}, where \textbfit{desert}
was typed when \textbfit{dessert} was intended. These errors will go undetected
by conventional spell checkers, which only flag words that are not found in a
predetermined word list or a dictionary.

\textit{\projectTitle{}} involves learning to characterize the linguistic
contexts in which different words such as \textbfit{dessert} and
\textbfit{desert}, tend to occur. The problem is that there is a multitude of
features one might use to characterize these contexts: features that test for
the presence of a particular word nearby the target word; features that test
the parts of speech around the target word; and so on. In general, the number
of features will range from a few hundred to over ten thousands. A good method
to filter out features is needed because even after filtering or removing noisy
features, we will still end up with a large feature space, while the target
concept (a context in which \textbfit{desert} can occur in this case) depends
only on a small subset. For this problem, the class of multiplicative weight
algorithms like Winnow Algorithm have been shown to have exceptionally good
theoretical properties.

In the work reported here, we present an algorithm that combines variants of
Winnow and applies it to the problem of \textit{\projectTitle{}}. We use the
Winnow Algorithm to find the best set of features which will then be used to
train linear classifiers which try to correct such contextual mismatches using
the learnt features.
